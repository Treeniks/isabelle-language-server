\documentclass[12pt,a4paper,fleqn]{article}
\usepackage{graphicx}
\usepackage{iman,extra,isar}
\usepackage{isabelle,isabellesym}
\usepackage{style}
\usepackage{pdfsetup}


\hyphenation{Isabelle}
\hyphenation{Isar}
\isadroptag{theory}

\title{\includegraphics[scale=0.5]{isabelle_logo}
  \\[4ex] Haskell-style type classes with Isabelle/Isar}
\author{\emph{Florian Haftmann}}

\begin{document}

\maketitle

\begin{abstract}
  \noindent This tutorial introduces Isar type classes, which 
  are a convenient mechanism for organizing specifications.
  Essentially, they combine an operational aspect (in the
  manner of Haskell) with a logical aspect, both managed uniformly.
\end{abstract}

\thispagestyle{empty}\clearpage

\pagenumbering{roman}
\clearfirst

\input{Classes.tex}

\begingroup
\bibliographystyle{plain} \small\raggedright\frenchspacing
\bibliography{manual}
\endgroup

\end{document}


%%% Local Variables: 
%%% mode: latex
%%% TeX-master: t
%%% End: 
