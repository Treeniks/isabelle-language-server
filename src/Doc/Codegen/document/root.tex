
\documentclass[12pt,a4paper,fleqn]{article}
\usepackage[T1]{fontenc}
\usepackage{graphicx}
\usepackage{tikz}\usetikzlibrary{shapes}\usetikzlibrary{arrows}
\usepackage{multirow}
\usepackage{iman,extra,isar}
\usepackage{isabelle,isabellesym}
\usepackage{style}
\usepackage{pdfsetup}

\hyphenation{Isabelle}
\hyphenation{Isar}
\isadroptag{theory}

\title{\includegraphics[scale=0.5]{isabelle_logo}
  \\[4ex] Code generation from Isabelle/HOL theories}
\author{\emph{Florian Haftmann}\\ with contributions by Lukas Bulwahn and Tobias Nipkow}

\begin{document}

\maketitle

\begin{abstract}
  \noindent This tutorial introduces the code generator facilities of Isabelle/HOL.
    They empower the user to turn HOL specifications into corresponding executable
    programs in the languages SML, OCaml, Haskell and Scala.
\end{abstract}

\thispagestyle{empty}\clearpage

\pagenumbering{roman}
\clearfirst

\input{Introduction.tex}
\input{Foundations.tex}
\input{Refinement.tex}
\input{Partial_Functions.tex}
\input{Inductive_Predicate.tex}
\input{Evaluation.tex}
\input{Computations.tex}
\input{Adaptation.tex}
\input{Further.tex}

\begingroup
\bibliographystyle{plain} \small\raggedright\frenchspacing
\bibliography{manual}
\endgroup

\end{document}


%%% Local Variables: 
%%% mode: latex
%%% TeX-master: t
%%% End: 
