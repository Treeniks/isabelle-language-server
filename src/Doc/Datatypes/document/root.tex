\documentclass[12pt,a4paper]{article} % fleqn
\usepackage[T1]{fontenc}
\usepackage{amsfonts}
\usepackage{amsmath}
\usepackage{cite}
\usepackage{enumitem}
\usepackage{footmisc}
\usepackage{graphicx}
\usepackage{iman}
\usepackage{extra}
\usepackage{isar}
\usepackage{isabelle}
\usepackage{isabellesym}
\usepackage{style}
\usepackage{pdfsetup}
\usepackage{railsetup}
\usepackage{framed}

\setcounter{secnumdepth}{3}
\setcounter{tocdepth}{3}

\renewcommand\_{\hbox{\textunderscore\kern-.05ex}}

\newbox\boxA
\setbox\boxA=\hbox{\ }
\parindent=4\wd\boxA

\newcommand\blankline{\vskip-.25\baselineskip}

\newenvironment{indentblock}{\list{}{\setlength{\leftmargin}{\parindent}}\item[]}{\endlist}

\newcommand{\keyw}[1]{\textbf{#1}}
\newcommand{\synt}[1]{\textit{#1}}
\newcommand{\hthm}[1]{\textbf{\textit{#1}}}

%\renewcommand{\isactrlsub}[1]{\/$\sb{\mathrm{#1}}$}
\renewcommand\isactrlsub[1]{\/$\sb{#1}$}
\renewcommand\isadigit[1]{\/\ensuremath{\mathrm{#1}}}
\renewcommand\isacharprime{\isamath{{'}\mskip-2mu}}
\renewcommand\isacharunderscore{\mbox{\_}}
\renewcommand\isacharunderscorekeyword{\mbox{\_}}
\renewcommand\isachardoublequote{\mbox{\upshape{``}}}
\renewcommand\isachardoublequoteopen{\mbox{\upshape{``}\kern.1ex}}
\renewcommand\isachardoublequoteclose{\/\kern.15ex\mbox{\upshape{''}}}
\renewcommand{\isasyminverse}{\isamath{{}^{-}}}

\hyphenation{isa-belle}

\isadroptag{theory}

\title{%\includegraphics[scale=0.5]{isabelle_hol} \\[4ex]
Defining (Co)datatypes and Primitively (Co)recursive Functions in Isabelle/HOL}
\author{Julian Biendarra, Jasmin Blanchette, \\
Martin Desharnais, Lorenz Panny, \\
Andrei Popescu, and Dmitriy Traytel}

\urlstyle{tt}

\begin{document}

\maketitle

\begin{sloppy}
\begin{abstract}
\noindent
This tutorial describes the definitional package for datatypes and codatatypes,
and for primitively recursive and corecursive functions, in Isabelle/HOL. The
following commands are provided:
\keyw{datatype}, \keyw{datatype_\allowbreak compat}, \keyw{prim\-rec}, \keyw{co\-data\-type},
\keyw{prim\-co\-rec}, \keyw{prim\-co\-recursive}, \keyw{bnf}, \keyw{lift_\allowbreak bnf},
\keyw{copy_\allowbreak bnf}, \keyw{bnf_\allowbreak axiom\-atization},
\keyw{print_\allowbreak bnfs}, and \keyw{free_\allowbreak constructors}.
\end{abstract}
\end{sloppy}

\tableofcontents

\input{Datatypes.tex}

\let\em=\sl
\bibliography{manual}{}
\bibliographystyle{abbrv}

\end{document}
