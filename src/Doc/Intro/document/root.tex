\documentclass[12pt,a4paper]{article}
\usepackage{graphicx,iman,extra,ttbox,proof,pdfsetup}

%% run    bibtex intro         to prepare bibliography
%% run    ../sedindex intro    to prepare index file
%prth *(\(.*\));          \1;      
%{\\out \(.*\)}          {\\out val it = "\1" : thm}

\title{\includegraphics[scale=0.5]{isabelle_logo} \\[4ex] Old Introduction to Isabelle}   
\author{{\em Lawrence C. Paulson}\\
        Computer Laboratory \\ University of Cambridge \\
        \texttt{lcp@cl.cam.ac.uk}\\[3ex] 
        With Contributions by Tobias Nipkow and Markus Wenzel
}
\makeindex

\underscoreoff

\setcounter{secnumdepth}{2} \setcounter{tocdepth}{2}

\sloppy
\binperiod     %%%treat . like a binary operator

\newcommand\qeq{\stackrel{?}{\equiv}}  %for disagreement pairs in unification
\newcommand{\nand}{\mathbin{\lnot\&}} 
\newcommand{\xor}{\mathbin{\#}}

\pagenumbering{roman} 
\begin{document}
\pagestyle{empty}
\begin{titlepage}
\maketitle 
\emph{Note}: this document is part of the earlier Isabelle documentation, 
which is largely superseded by the Isabelle/HOL
\emph{Tutorial}~\cite{isa-tutorial}. It describes the old-style theory 
syntax and shows how to conduct proofs using the 
ML top level. This style of interaction is largely obsolete:
most Isabelle proofs are now written using the Isar 
language and the Proof General interface. However, this paper contains valuable 
information that is not available elsewhere. Its examples are based 
on first-order logic rather than higher-order logic.

\thispagestyle{empty}
\vfill
{\small Copyright \copyright{} \number\year{} by Lawrence C. Paulson}
\end{titlepage}

\pagestyle{headings}
\part*{Preface}
\index{Isabelle!overview} \index{Isabelle!object-logics supported}
Isabelle~\cite{paulson-natural,paulson-found,paulson700} is a generic theorem
prover.  It has been instantiated to support reasoning in several
object-logics:
\begin{itemize}
\item first-order logic, constructive and classical versions
\item higher-order logic, similar to that of Gordon's {\sc
hol}~\cite{mgordon-hol}
\item Zermelo-Fraenkel set theory~\cite{suppes72}
\item an extensional version of Martin-L\"of's Type Theory~\cite{nordstrom90}
\item the classical first-order sequent calculus, {\sc lk}
\item the modal logics $T$, $S4$, and $S43$
\item the Logic for Computable Functions~\cite{paulson87}
\end{itemize}
A logic's syntax and inference rules are specified declaratively; this
allows single-step proof construction.  Isabelle provides control
structures for expressing search procedures.  Isabelle also provides
several generic tools, such as simplifiers and classical theorem provers,
which can be applied to object-logics.

Isabelle is a large system, but beginners can get by with a small
repertoire of commands and a basic knowledge of how Isabelle works.  
The Isabelle/HOL \emph{Tutorial}~\cite{isa-tutorial} describes how to get started. Advanced Isabelle users will benefit from some
knowledge of Standard~\ML{}, because Isabelle is written in \ML{};
\index{ML}
if you are prepared to writing \ML{}
code, you can get Isabelle to do almost anything.  My book
on~\ML{}~\cite{paulson-ml2} covers much material connected with Isabelle,
including a simple theorem prover.  Users must be familiar with logic as
used in computer science; there are many good
texts~\cite{galton90,reeves90}.

\index{LCF}
{\sc lcf}, developed by Robin Milner and colleagues~\cite{mgordon79}, is an
ancestor of {\sc hol}, Nuprl, and several other systems.  Isabelle borrows
ideas from {\sc lcf}: formulae are~\ML{} values; theorems belong to an
abstract type; tactics and tacticals support backward proof.  But {\sc lcf}
represents object-level rules by functions, while Isabelle represents them
by terms.  You may find my other writings~\cite{paulson87,paulson-handbook}
helpful in understanding the relationship between {\sc lcf} and Isabelle.

\index{Isabelle!release history} Isabelle was first distributed in 1986.
The 1987 version introduced a higher-order meta-logic with an improved
treatment of quantifiers.  The 1988 version added limited polymorphism and
support for natural deduction.  The 1989 version included a parser and
pretty printer generator.  The 1992 version introduced type classes, to
support many-sorted and higher-order logics.  The 1994 version introduced
greater support for theories.  The most important recent change is the
introduction of the Isar proof language, thanks to Markus Wenzel.  
Isabelle is still under
development and will continue to change.

\subsubsection*{Overview} 
This manual consists of three parts.  Part~I discusses the Isabelle's
foundations.  Part~II, presents simple on-line sessions, starting with
forward proof.  It also covers basic tactics and tacticals, and some
commands for invoking them.  Part~III contains further examples for users
with a bit of experience.  It explains how to derive rules define theories,
and concludes with an extended example: a Prolog interpreter.

Isabelle's Reference Manual and Object-Logics manual contain more details.
They assume familiarity with the concepts presented here.


\subsubsection*{Acknowledgements} 
Tobias Nipkow contributed most of the section on defining theories.
Stefan Berghofer, Sara Kalvala and Viktor Kuncak suggested improvements.

Tobias Nipkow has made immense contributions to Isabelle, including the parser
generator, type classes, and the simplifier.  Carsten Clasohm and Markus
Wenzel made major contributions; Sonia Mahjoub and Karin Nimmermann also
helped.  Isabelle was developed using Dave Matthews's Standard~{\sc ml}
compiler, Poly/{\sc ml}.  Many people have contributed to Isabelle's standard
object-logics, including Martin Coen, Philippe de Groote, Philippe No\"el.
The research has been funded by the EPSRC (grants GR/G53279, GR/H40570,
GR/K57381, GR/K77051, GR/M75440) and by ESPRIT (projects 3245: Logical
Frameworks, and 6453: Types), and by the DFG Schwerpunktprogramm
\emph{Deduktion}.

\newpage
\pagestyle{plain} \tableofcontents 
\newpage

\newfont{\sanssi}{cmssi12}
\vspace*{2.5cm}
\begin{quote}
\raggedleft
{\sanssi
You can only find truth with logic\\
if you have already found truth without it.}\\
\bigskip

G.K. Chesterton, {\em The Man who was Orthodox}
\end{quote}

\clearfirst  \pagestyle{headings}
\input{foundations}
\input{getting}
\input{advanced}

\bibliographystyle{plain} \small\raggedright\frenchspacing
\bibliography{manual}

\printindex
\end{document}
