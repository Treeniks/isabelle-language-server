\documentclass[a4paper,12pt]{article}
\usepackage[T1]{fontenc}
\usepackage{amsmath}
\usepackage{amssymb}
\usepackage{color}
\usepackage{footmisc}
\usepackage{graphicx}
%\usepackage{mathpazo}
\usepackage{multicol}
\usepackage{stmaryrd}
%\usepackage[scaled=.85]{beramono}
\usepackage{isabelle,iman,pdfsetup}

\newcommand\download{\url{https://isabelle.in.tum.de/components/}}

\let\oldS=\S
\def\S{\oldS\,}

\def\qty#1{\ensuremath{\left<\mathit{#1\/}\right>}}
\def\qtybf#1{$\mathbf{\left<\textbf{\textit{#1\/}}\right>}$}

\newcommand\const[1]{\textsf{#1}}

%\oddsidemargin=4.6mm
%\evensidemargin=4.6mm
%\textwidth=150mm
%\topmargin=4.6mm
%\headheight=0mm
%\headsep=0mm
%\textheight=234mm

\def\Colon{\mathord{:\mkern-1.5mu:}}
%\def\lbrakk{\mathopen{\lbrack\mkern-3.25mu\lbrack}}
%\def\rbrakk{\mathclose{\rbrack\mkern-3.255mu\rbrack}}
\def\lparr{\mathopen{(\mkern-4mu\mid}}
\def\rparr{\mathclose{\mid\mkern-4mu)}}

\def\unk{{?}}
\def\undef{(\lambda x.\; \unk)}
%\def\unr{\textit{others}}
\def\unr{\ldots}
\def\Abs#1{\hbox{\rm{\guillemetleft}}{\,#1\,}\hbox{\rm{\guillemetright}}}
\def\Q{{\smash{\lower.2ex\hbox{$\scriptstyle?$}}}}

\urlstyle{tt}

\renewcommand\_{\hbox{\textunderscore\kern-.05ex}}

\hyphenation{Isa-belle super-posi-tion zipper-posi-tion}

\begin{document}

%%% TYPESETTING
%\renewcommand\labelitemi{$\bullet$}
\renewcommand\labelitemi{\raise.065ex\hbox{\small\textbullet}}

\title{\includegraphics[scale=0.5]{isabelle_logo} \\[4ex]
Hammering Away \\[\smallskipamount]
\Large A User's Guide to Sledgehammer for Isabelle/HOL}
\author{\hbox{} \\
Jasmin Blanchette \\
{\normalsize Institut f\"ur Informatik, Technische Universit\"at M\"unchen} \\[4\smallskipamount]
{\normalsize with contributions from} \\[4\smallskipamount]
Martin Desharnais \\
{\normalsize Forschungsinstitut CODE, Universit\"at der Bundeswehr M\"unchen}  \\[4\smallskipamount]
Lawrence C. Paulson \\
{\normalsize Computer Laboratory, University of Cambridge} \\
\hbox{}}

\maketitle

\tableofcontents

\setlength{\parskip}{.7em plus .2em minus .1em}
\setlength{\parindent}{0pt}
\setlength{\abovedisplayskip}{\parskip}
\setlength{\abovedisplayshortskip}{.9\parskip}
\setlength{\belowdisplayskip}{\parskip}
\setlength{\belowdisplayshortskip}{.9\parskip}

% general-purpose enum environment with correct spacing
\newenvironment{enum}%
    {\begin{list}{}{%
        \setlength{\topsep}{.1\parskip}%
        \setlength{\partopsep}{.1\parskip}%
        \setlength{\itemsep}{\parskip}%
        \advance\itemsep by-\parsep}}
    {\end{list}}

\def\pre{\begingroup\vskip0pt plus1ex\advance\leftskip by\leftmargin
\advance\rightskip by\leftmargin}
\def\post{\vskip0pt plus1ex\endgroup}

\def\prew{\pre\advance\rightskip by-\leftmargin}
\def\postw{\post}


\section{Introduction}
\label{introduction}

Sledgehammer is a tool that applies automatic theorem provers (ATPs)
and satisfiability-modulo-theories (SMT) solvers on the current goal.%
\footnote{The distinction between ATPs and SMT solvers is convenient but mostly
historical.}
%
The supported ATPs include agsyHOL \cite{agsyHOL}, Alt-Ergo \cite{alt-ergo}, E
\cite{schulz-2019}, iProver \cite{korovin-2009}, LEO-II \cite{leo2}, Leo-III
\cite{leo3}, Satallax \cite{satallax}, SPASS \cite{weidenbach-et-al-2009},
Vampire \cite{riazanov-voronkov-2002}, Waldmeister \cite{waldmeister}, and
Zipperposition \cite{cruanes-2014}. The ATPs are run either locally or remotely
via the System\-On\-TPTP web service \cite{sutcliffe-2000}. The supported SMT
solvers are CVC4 \cite{cvc4}, veriT \cite{bouton-et-al-2009}, and Z3
\cite{de-moura-2008}. These are always run locally.

The problem passed to the external provers (or solvers) consists of your current
goal together with a heuristic selection of hundreds of facts (theorems) from the
current theory context, filtered by relevance.

The result of a successful proof search is some source text that typically
reconstructs the proof within Isabelle. For ATPs, the reconstructed proof
typically relies on the general-purpose \textit{metis} proof method, which
integrates the Metis ATP in Isabelle/HOL with explicit inferences going through
the kernel. Thus its results are correct by construction.

For Isabelle/jEdit users, Sledgehammer provides an automatic mode that can be
enabled via the ``Auto Sledgehammer'' option under ``Plugins > Plugin Options >
Isabelle > General.'' In this mode, a reduced version of Sledgehammer is run on
every newly entered theorem for a few seconds.

\newbox\boxA
\setbox\boxA=\hbox{\texttt{NOSPAM}}

\newcommand\authoremail{\texttt{jasmin.blan{\color{white}NOSPAM}\kern-\wd\boxA{}chette@\allowbreak
google.\allowbreak com}}

To run Sledgehammer, you must make sure that the theory \textit{Sledgehammer} is
imported---this is rarely a problem in practice since it is part of
\textit{Main}. Examples of Sledgehammer use can be found in the
\texttt{src/HOL/Metis\_Examples} directory.  Comments and bug reports
concerning Sledgehammer or this manual should be directed to the author at
\authoremail.


\section{Installation}
\label{installation}

Sledgehammer is part of Isabelle, so you do not need to install it. However, it
relies on third-party automatic provers (ATPs and SMT solvers).

Among the ATPs, agsyHOL, Alt-Ergo, E, LEO-II, Leo-III, Satallax, SPASS, Vampire,
and Zipperposition can be run locally; in addition, agsyHOL, Alt-Ergo, E,
iProver, LEO-II, Leo-III, Satallax, Vampire, Waldmeister, and Zipperposition are
available remotely via System\-On\-TPTP \cite{sutcliffe-2000}. The SMT solvers
CVC4, veriT, and Z3 can be run locally.

There are three main ways to install automatic provers on your machine:

\begin{sloppy}
\begin{enum}
\item[\labelitemi] If you installed an official Isabelle package, it should
already include properly set up executables for CVC4, E, SPASS, Vampire, veriT,
and Z3, ready to use.

\item[\labelitemi] Alternatively, you can download the Isabelle-aware CVC4, E,
SPASS, Vampire, veriT, and Z3 binary packages from \download. Extract the
archives, then add a line to your \texttt{\$ISABELLE\_HOME\_USER\slash
etc\slash components}%
\footnote{The variable \texttt{\$ISABELLE\_HOME\_USER} is set by Isabelle at
startup. Its value can be retrieved by executing \texttt{isabelle}
\texttt{getenv} \texttt{ISABELLE\_HOME\_USER} on the command line.}
file with the absolute path to the prover. For example, if the
\texttt{components} file does not exist yet and you extracted SPASS to
\texttt{/usr/local/spass-3.8ds}, create it with the single line

\prew
\texttt{/usr/local/spass-3.8ds}
\postw

in it.

\item[\labelitemi] If you prefer to build agsyHOL, Alt-Ergo, E, LEO-II,
Leo-III, or Satallax manually, set the environment variable
\texttt{AGSYHOL\_HOME}, \texttt{E\_HOME}, \texttt{LEO2\_HOME},
\texttt{LEO3\_HOME}, or \texttt{SATALLAX\_HOME}
to the directory that contains the \texttt{agsyHOL},
\texttt{eprover} (and/or \texttt{eproof} or \texttt{eproof\_ram}),
\texttt{leo}, \texttt{leo3}, or \texttt{satallax} executable;
for Alt-Ergo, set the environment variable \texttt{WHY3\_HOME} to the
directory that contains the \texttt{why3} executable. Ideally, you
should also set \texttt{E\_VERSION}, \texttt{LEO2\_VERSION},
\texttt{LEO3\_VERSION}, or \texttt{SATALLAX\_VERSION} to the prover's version
number (e.g., ``3.6'').

Similarly, if you want to install CVC4, veriT, or Z3, set the environment
variable \texttt{CVC4\_\allowbreak SOLVER}, \texttt{ISABELLE\_\allowbreak VERIT},
or \texttt{Z3\_SOLVER} to the complete path of the executable, \emph{including
the file name}. Ideally, also set \texttt{CVC4\_VERSION},
\texttt{VERIT\_VERSION}, or \texttt{Z3\_VERSION} to the solver's version number
(e.g., ``4.4.0'').
\end{enum}
\end{sloppy}

To check whether the provers are successfully installed, try out the example
in \S\ref{first-steps}. If the remote versions of any of these provers is used
(identified by the prefix ``\textit{remote\_\/}''), or if the local versions
fail to solve the easy goal presented there, something must be wrong with the
installation.


\section{First Steps}
\label{first-steps}

To illustrate Sledgehammer in context, let us start a theory file and
attempt to prove a simple lemma:

\prew
\textbf{theory}~\textit{Scratch} \\
\textbf{imports}~\textit{Main} \\
\textbf{begin} \\[2\smallskipamount]
%
\textbf{lemma} ``$[a] = [b] \,\Longrightarrow\, a = b$'' \\
\textbf{sledgehammer}
\postw

Instead of issuing the \textbf{sledgehammer} command, you can also use the
Sledgehammer panel in Isabelle/jEdit. Sledgehammer might produce something like
the following output after a few seconds:

\prew
\slshape
Proof found\ldots \\
``\textit{e\/}'': Try this: \textbf{by} \textit{simp} (0.3 ms) \\
%
``\textit{cvc4\/}'': Try this: \textbf{by} \textit{simp} (0.4 ms) \\
%
``\textit{z3\/}'': Try this: \textbf{by} \textit{simp} (0.5 ms) \\
%
``\textit{vampire\/}'': Try this: \textbf{by} \textit{simp} (0.3 ms)
%
\postw

Sledgehammer ran CVC4, E, Vampire, and Z3 in parallel. This list may vary
depending on which provers are installed and how many processor cores are
available.

For each successful prover, Sledgehammer gives a one-line Isabelle proof. Rough
timings are shown in parentheses, indicating how fast the call is. You can
click the proof to insert it into the theory text.

In addition, you can ask Sledgehammer for an Isar text proof by enabling the
\textit{isar\_proofs} option (\S\ref{output-format}):

\prew
\textbf{sledgehammer} [\textit{isar\_proofs}]
\postw

When Isar proof construction is successful, it can yield proofs that are more
readable and also faster than \textit{metis} or \textit{smt} one-line
proofs. This feature is experimental.


\section{Hints}
\label{hints}

This section presents a few hints that should help you get the most out of
Sledgehammer. Frequently asked questions are answered in
\S\ref{frequently-asked-questions}.

%\newcommand\point[1]{\medskip\par{\sl\bfseries#1}\par\nopagebreak}
\newcommand\point[1]{\subsection{\emph{#1}}}


\point{Presimplify the goal}

For best results, first simplify your problem by calling \textit{auto} or at
least \textit{safe} followed by \textit{simp\_all}. The SMT solvers provide
arithmetic decision procedures, but the ATPs typically do not (or if they do,
Sledgehammer does not use it yet). Apart from Waldmeister, they are not
particularly good at heavy rewriting, but because they regard equations as
undirected, they often prove theorems that require the reverse orientation of a
\textit{simp} rule. Higher-order problems can be tackled, but the success rate
is better for first-order problems. Hence, you may get better results if you
first simplify the problem to remove higher-order features.


\point{Familiarize yourself with the main options}

Sledgehammer's options are fully documented in \S\ref{command-syntax}. Many of
the options are very specialized, but serious users of the tool should at least
familiarize themselves with the following options:

\begin{enum}
\item[\labelitemi] \textbf{\textit{provers}} (\S\ref{mode-of-operation}) specifies
the automatic provers (ATPs and SMT solvers) that should be run whenever
Sledgehammer is invoked (e.g., ``\textit{provers}~= \textit{cvc4 e spass
vampire\/}''). For convenience, you can omit ``\textit{provers}~=''
and simply write the prover names as a space-separated list (e.g., ``\textit{cvc4 e
spass vampire\/}'').

\item[\labelitemi] \textbf{\textit{max\_facts}} (\S\ref{relevance-filter})
specifies the maximum number of facts that should be passed to the provers. By
default, the value is prover-dependent but varies between about 50 and 1000. If
the provers time out, you can try lowering this value to, say, 25 or 50 and see
if that helps.

\item[\labelitemi] \textbf{\textit{isar\_proofs}} (\S\ref{output-format}) specifies
that Isar proofs should be generated, in addition to one-line \textit{metis} or
\textit{smt} proofs. The length of the Isar proofs can be controlled by setting
\textit{compress} (\S\ref{output-format}).

\item[\labelitemi] \textbf{\textit{timeout}} (\S\ref{timeouts}) controls the
provers' time limit. It is set to 30 seconds by default.
\end{enum}

Options can be set globally using \textbf{sledgehammer\_params}
(\S\ref{command-syntax}). The command also prints the list of all available
options with their current value. Fact selection can be influenced by specifying
``$(\textit{add}{:}~\textit{my\_facts})$'' after the \textbf{sledgehammer} call
to ensure that certain facts are included, or simply ``$(\textit{my\_facts})$''
to force Sledgehammer to run only with $\textit{my\_facts}$ (and any facts
chained into the goal).


\section{Frequently Asked Questions}
\label{frequently-asked-questions}

This sections answers frequently (and infrequently) asked questions about
Sledgehammer. It is a good idea to skim over it now even if you do not have any
questions at this stage. And if you have any further questions not listed here,
send them to the author at \authoremail.


\point{Which facts are passed to the automatic provers?}

Sledgehammer heuristically selects a few hundred relevant lemmas from the
currently loaded libraries. The component that performs this selection is
called \emph{relevance filter} (\S\ref{relevance-filter}).

\begin{enum}
\item[\labelitemi]
The traditional relevance filter, \emph{MePo}
(\underline{Me}ng--\underline{Pau}lson), assigns a score to every available
fact (lemma, theorem, definition, or axiom) based upon how many constants that
fact shares with the conjecture. This process iterates to include facts
relevant to those just accepted. The constants are weighted to give unusual
ones greater significance. MePo copes best when the conjecture contains some
unusual constants; if all the constants are common, it is unable to
discriminate among the hundreds of facts that are picked up. The filter is also
memoryless: It has no information about how many times a particular fact has
been used in a proof, and it cannot learn.

\item[\labelitemi]
An alternative to MePo is \emph{MaSh} (\underline{Ma}chine Learner for
\underline{S}ledge\underline{h}ammer). It applies machine learning to the
problem of finding relevant facts.

\item[\labelitemi] The \emph{MeSh} filter combines MePo and MaSh. This is
the default.
\end{enum}

The number of facts included in a problem varies from prover to prover, since
some provers get overwhelmed more easily than others. You can show the number of
facts given using the \textit{verbose} option (\S\ref{output-format}) and the
actual facts using \textit{debug} (\S\ref{output-format}).

Sledgehammer is good at finding short proofs combining a handful of existing
lemmas. If you are looking for longer proofs, you must typically restrict the
number of facts, by setting the \textit{max\_facts} option
(\S\ref{relevance-filter}) to, say, 25 or 50.

You can also influence which facts are actually selected in a number of ways. If
you simply want to ensure that a fact is included, you can specify it using the
``$(\textit{add}{:}~\textit{my\_facts})$'' syntax. For example:
%
\prew
\textbf{sledgehammer} (\textit{add}: \textit{hd.simps} \textit{tl.simps})
\postw
%
The specified facts then replace the least relevant facts that would otherwise be
included; the other selected facts remain the same.
If you want to direct the selection in a particular direction, you can specify
the facts via \textbf{using}:
%
\prew
\textbf{using} \textit{hd.simps} \textit{tl.simps} \\
\textbf{sledgehammer}
\postw
%
The facts are then more likely to be selected than otherwise, and if they are
selected at iteration $j$ they also influence which facts are selected at
iterations $j + 1$, $j + 2$, etc. To give them even more weight, try
%
\prew
\textbf{using} \textit{hd.simps} \textit{tl.simps} \\
\textbf{apply}~\textbf{--} \\
\textbf{sledgehammer}
\postw


\point{Why does Metis fail to reconstruct the proof?}

There are many reasons. If Metis runs seemingly forever, that is a sign that the
proof is too difficult for it. Metis's search is complete for first-order logic
with equality, so if the proof was found by a superposition-based ATP such as
E, SPASS, or Vampire, Metis should eventually find it, but that is little
consolation.

In some rare cases, \textit{metis} fails fairly quickly, and you get the error
message ``One-line proof reconstruction failed.'' This indicates that
Sledgehammer determined that the goal is provable, but the proof is, for
technical reasons, beyond \textit{metis}'s power. You can then try again with
the \textit{strict} option (\S\ref{problem-encoding}).

If the goal is actually unprovable and you did not specify an unsound encoding
using \textit{type\_enc} (\S\ref{problem-encoding}), this is a bug, and you are
strongly encouraged to report this to the author at \authoremail.


\point{What are the \textit{full\_types}, \textit{no\_types}, and \\
\textit{mono\_tags} arguments to Metis?}

The \textit{metis}~(\textit{full\_types}) proof method
and its cousin \textit{metis}~(\textit{mono\_tags}) are fully-typed
versions of Metis. It is somewhat slower than \textit{metis}, but the proof
search is fully typed, and it also includes more powerful rules such as the
axiom ``$x = \const{True} \mathrel{\lor} x = \const{False}$'' for reasoning in
higher-order places (e.g., in set comprehensions). The method is tried as a
fallback when \textit{metis} fails, and it is sometimes generated by
Sledgehammer instead of \textit{metis} if the proof obviously requires type
information or if \textit{metis} failed when Sledgehammer preplayed the proof.
%
At the other end of the soundness spectrum, \textit{metis} (\textit{no\_types})
uses no type information at all during the proof search, which is more efficient
but often fails. Calls to \textit{metis} (\textit{no\_types}) are occasionally
generated by Sledgehammer.
%
See the \textit{type\_enc} option (\S\ref{problem-encoding}) for details.

Incidentally, if you ever see warnings such as

\prew
\slshape
Metis: Falling back on ``\textit{metis} (\textit{full\_types})''
\postw

for a successful \textit{metis} proof, you can advantageously pass the
\textit{full\_types} option to \textit{metis} directly.


\point{And what are the \textit{lifting} and \textit{opaque\_lifting} \\
arguments to Metis?}

Orthogonally to the encoding of types, it is important to choose an appropriate
translation of $\lambda$-abstractions. Metis supports three translation
schemes, in decreasing order of power: Curry combinators (the default),
$\lambda$-lifting, and a ``hiding'' scheme that disables all reasoning under
$\lambda$-abstractions. The more powerful schemes also give the automatic
provers more rope to hang themselves. See the \textit{lam\_trans} option
(\S\ref{problem-encoding}) for details.


\point{Are the generated proofs minimal?}

Automatic provers frequently use many more facts than are necessary.
Sledgehammer includes a proof minimization tool that takes a set of facts returned by
a given prover and repeatedly calls a prover or proof method with subsets of
those facts to find a minimal set. Reducing the number of facts typically helps
reconstruction, while decluttering the proof scripts.


\point{A strange error occurred---what should I do?}

Sledgehammer tries to give informative error messages. Please report any strange
error to the author at \authoremail.


\point{Auto can solve it---why not Sledgehammer?}

Problems can be easy for \textit{auto} and difficult for automatic provers, but
the reverse is also true, so do not be discouraged if your first attempts fail.
Because the system refers to all theorems known to Isabelle, it is particularly
suitable when your goal has a short proof but requires lemmas that you do not
know about.


\point{Why are there so many options?}

Sledgehammer's philosophy is that it should work out of the box, without user
guidance. Most of the options are meant to be used by the Sledgehammer
developers for experiments.


\section{Command Syntax}
\label{command-syntax}

\subsection{Sledgehammer}
\label{sledgehammer}

Sledgehammer can be invoked at any point when there is an open goal by entering
the \textbf{sledgehammer} command in the theory file. Its general syntax is as
follows:

\prew
\textbf{sledgehammer} \qty{subcommand}$^?$ \qty{options}$^?$ \qty{facts\_override}$^?$ \qty{num}$^?$
\postw

In the general syntax, the \qty{subcommand} may be any of the following:

\begin{enum}
\item[\labelitemi] \textbf{\textit{run} (the default):} Runs Sledgehammer on
subgoal number \qty{num} (1 by default), with the given options and facts.

\item[\labelitemi] \textbf{\textit{supported\_provers}:} Prints the list of
automatic provers supported by Sledgehammer. See \S\ref{installation} and
\S\ref{mode-of-operation} for more information on how to install automatic
provers.

\item[\labelitemi] \textbf{\textit{refresh\_tptp}:} Refreshes the list of remote
ATPs available at System\-On\-TPTP \cite{sutcliffe-2000}.
\end{enum}

In addition, the following subcommands provide finer control over machine
learning with MaSh:

\begin{enum}
\item[\labelitemi] \textbf{\textit{unlearn}:} Resets MaSh, erasing any
persistent state.

\item[\labelitemi] \textbf{\textit{learn\_isar}:} Invokes MaSh on the current
theory to process all the available facts, learning from their Isabelle/Isar
proofs. This happens automatically at Sledgehammer invocations if the
\textit{learn} option (\S\ref{relevance-filter}) is enabled.

\item[\labelitemi] \textbf{\textit{learn\_prover}:} Invokes MaSh on the current
theory to process all the available facts, learning from proofs generated by
automatic provers. The prover to use and its timeout can be set using the
\textit{prover} (\S\ref{mode-of-operation}) and \textit{timeout}
(\S\ref{timeouts}) options. It is recommended to perform learning using a
first-order ATP (such as E, SPASS, and Vampire) as opposed to a
higher-order ATP or an SMT solver.

\item[\labelitemi] \textbf{\textit{relearn\_isar}:} Same as \textit{unlearn}
followed by \textit{learn\_isar}.

\item[\labelitemi] \textbf{\textit{relearn\_prover}:} Same as \textit{unlearn}
followed by \textit{learn\_prover}.
\end{enum}

Sledgehammer's behavior can be influenced by various \qty{options}, which can be
specified in brackets after the \textbf{sledgehammer} command. The
\qty{options} are a list of key--value pairs of the form ``[$k_1 = v_1,
\ldots, k_n = v_n$]''. For Boolean options, ``= \textit{true\/}'' is optional.
For example:

\prew
\textbf{sledgehammer} [\textit{isar\_proofs}, \,\textit{timeout} = 120]
\postw

Default values can be set using \textbf{sledgehammer\_\allowbreak params}:

\prew
\textbf{sledgehammer\_params} \qty{options}
\postw

The supported options are described in \S\ref{option-reference}.

The \qty{facts\_override} argument lets you alter the set of facts that go
through the relevance filter. It may be of the form ``(\qty{facts})'', where
\qty{facts} is a space-separated list of Isabelle facts (theorems, local
assumptions, etc.), in which case the relevance filter is bypassed and the given
facts are used. It may also be of the form ``(\textit{add}:\ \qty{facts\/_{\mathrm{1}}})'',
``(\textit{del}:\ \qty{facts\/_{\mathrm{2}}})'', or ``(\textit{add}:\ \qty{facts\/_{\mathrm{1}}}\
\textit{del}:\ \qty{facts\/_{\mathrm{2}}})'', where the relevance filter is instructed to
proceed as usual except that it should consider \qty{facts\/_{\mathrm{1}}}
highly-relevant and \qty{facts\/_{\mathrm{2}}} fully irrelevant.

If you use Isabelle/jEdit, Sledgehammer also provides an automatic mode that can
be enabled via the ``Auto Sledgehammer'' option under ``Plugins > Plugin Options
> Isabelle > General.'' For automatic runs, only the first prover set using
\textit{provers} (\S\ref{mode-of-operation}) is considered (typically E),
\textit{slice} (\S\ref{mode-of-operation}) is disabled,
fewer facts are
passed to the prover, \textit{fact\_filter} (\S\ref{relevance-filter}) is set to
\textit{mepo}, \textit{strict} (\S\ref{problem-encoding}) is enabled,
\textit{verbose} (\S\ref{output-format}) and \textit{debug}
(\S\ref{output-format}) are disabled, and \textit{timeout} (\S\ref{timeouts}) is
superseded by the ``Auto Time Limit'' option in jEdit. Sledgehammer's output is
also more concise.


\subsection{Metis}
\label{metis}

The \textit{metis} proof method has the syntax

\prew
\textbf{\textit{metis}}~(\qty{options})${}^?$~\qty{facts}${}^?$
\postw

where \qty{facts} is a list of arbitrary facts and \qty{options} is a
comma-separated list consisting of at most one $\lambda$ translation scheme
specification with the same semantics as Sledgehammer's \textit{lam\_trans}
option (\S\ref{problem-encoding}) and at most one type encoding specification
with the same semantics as Sledgehammer's \textit{type\_enc} option
(\S\ref{problem-encoding}).
%
The supported $\lambda$ translation schemes are \textit{opaque\_lifting},
\textit{lifting}, and \textit{combs} (the default).
%
All the untyped type encodings listed in \S\ref{problem-encoding} are supported.
For convenience, the following aliases are provided:
\begin{enum}
\item[\labelitemi] \textbf{\textit{full\_types}:} Alias for \textit{poly\_guards\_query}.
\item[\labelitemi] \textbf{\textit{partial\_types}:} Alias for \textit{poly\_args}.
\item[\labelitemi] \textbf{\textit{no\_types}:} Alias for \textit{erased}.
\end{enum}


\section{Option Reference}
\label{option-reference}

\def\defl{\{}
\def\defr{\}}

\def\flushitem#1{\item[]\noindent\kern-\leftmargin \textbf{#1}}
\def\optrueonly#1{\flushitem{\textit{#1} $\bigl[$= \textit{true}$\bigr]$\enskip}\nopagebreak\\[\parskip]}
\def\optrue#1#2{\flushitem{\textit{#1} $\bigl[$= \qtybf{bool}$\bigr]$\enskip \defl\textit{true}\defr\hfill (neg.: \textit{#2})}\nopagebreak\\[\parskip]}
\def\opfalse#1#2{\flushitem{\textit{#1} $\bigl[$= \qtybf{bool}$\bigr]$\enskip \defl\textit{false}\defr\hfill (neg.: \textit{#2})}\nopagebreak\\[\parskip]}
\def\opsmart#1#2{\flushitem{\textit{#1} $\bigl[$= \qtybf{smart\_bool}$\bigr]$\enskip \defl\textit{smart}\defr\hfill (neg.: \textit{#2})}\nopagebreak\\[\parskip]}
\def\opsmartx#1#2{\flushitem{\textit{#1} $\bigl[$= \qtybf{smart\_bool}$\bigr]$\enskip \defl\textit{smart}\defr\\\hbox{}\hfill (neg.: \textit{#2})}\nopagebreak\\[\parskip]}
\def\opnodefault#1#2{\flushitem{\textit{#1} = \qtybf{#2}} \nopagebreak\\[\parskip]}
\def\opnodefaultbrk#1#2{\flushitem{$\bigl[$\textit{#1} =$\bigr]$ \qtybf{#2}} \nopagebreak\\[\parskip]}
\def\opdefault#1#2#3{\flushitem{\textit{#1} = \qtybf{#2}\enskip \defl\textit{#3}\defr} \nopagebreak\\[\parskip]}
\def\oparg#1#2#3{\flushitem{\textit{#1} \qtybf{#2} = \qtybf{#3}} \nopagebreak\\[\parskip]}
\def\opargbool#1#2#3{\flushitem{\textit{#1} \qtybf{#2} $\bigl[$= \qtybf{bool}$\bigr]$\hfill (neg.: \textit{#3})}\nopagebreak\\[\parskip]}
\def\opargboolorsmart#1#2#3{\flushitem{\textit{#1} \qtybf{#2} $\bigl[$= \qtybf{smart\_bool}$\bigr]$\hfill (neg.: \textit{#3})}\nopagebreak\\[\parskip]}

Sledgehammer's options are categorized as follows:\ mode of operation
(\S\ref{mode-of-operation}), problem encoding (\S\ref{problem-encoding}),
relevance filter (\S\ref{relevance-filter}), output format
(\S\ref{output-format}), regression testing (\S\ref{regression-testing}),
and timeouts (\S\ref{timeouts}).

The descriptions below refer to the following syntactic quantities:

\begin{enum}
\item[\labelitemi] \qtybf{string}: A string.
\item[\labelitemi] \qtybf{bool\/}: \textit{true} or \textit{false}.
\item[\labelitemi] \qtybf{smart\_bool\/}: \textit{true}, \textit{false}, or
\textit{smart}.
\item[\labelitemi] \qtybf{int\/}: An integer.
\item[\labelitemi] \qtybf{float}: A floating-point number (e.g., 2.5 or 60)
expressing a number of seconds.
\item[\labelitemi] \qtybf{float\_pair\/}: A pair of floating-point numbers
(e.g., 0.6 0.95).
\item[\labelitemi] \qtybf{smart\_int\/}: An integer or \textit{smart}.
\end{enum}

Default values are indicated in curly brackets (\textrm{\{\}}). Boolean options
have a negative counterpart (e.g., \textit{minimize} vs.\
\textit{dont\_minimize}). When setting Boolean options or their negative
counterparts, ``= \textit{true\/}'' may be omitted.


\subsection{Mode of Operation}
\label{mode-of-operation}

\begin{enum}
\opnodefaultbrk{provers}{string}
Specifies the automatic provers to use as a space-separated list (e.g.,
``\textit{cvc4}~\textit{e}~\textit{spass}~\textit{vampire\/}'').
Provers can be run locally or remotely; see \S\ref{installation} for
installation instructions.

The following local provers are supported:

\begin{sloppy}
\begin{enum}
\item[\labelitemi] \textbf{\textit{agsyhol}:} agsyHOL is an automatic
higher-order prover developed by Fredrik Lindblad \cite{agsyHOL}. To use
agsyHOL, set the environment variable \texttt{AGSYHOL\_HOME} to the directory
that contains the \texttt{agsyHOL} executable.

\item[\labelitemi] \textbf{\textit{alt\_ergo}:} Alt-Ergo is a polymorphic
ATP developed by Bobot et al.\ \cite{alt-ergo}.
It supports the TPTP polymorphic typed first-order format (TF1) via Why3
\cite{why3}. To use Alt-Ergo, set the environment variable \texttt{WHY3\_HOME}
to the directory that contains the \texttt{why3} executable. Sledgehammer
requires Alt-Ergo 0.95.2 and Why3 0.83.

\item[\labelitemi] \textbf{\textit{cvc4}:} CVC4 is an SMT solver developed by
Clark Barrett, Cesare Tinelli, and their colleagues \cite{cvc4}. To use CVC4,
set the environment variable \texttt{CVC4\_SOLVER} to the complete path of the
executable, including the file name, or install the prebuilt CVC4 package from
\download.

\item[\labelitemi] \textbf{\textit{e}:} E is a first-order resolution prover
developed by Stephan Schulz \cite{schulz-2019}. To use E, set the environment
variable \texttt{E\_HOME} to the directory that contains the \texttt{eproof}
executable and \texttt{E\_VERSION} to the version number (e.g., ``1.8''), or
install the prebuilt E package from \download.

\item[\labelitemi] \textbf{\textit{iprover}:} iProver is a pure
instantiation-based prover developed by Konstantin Korovin \cite{korovin-2009}.
To use iProver, set the environment variable \texttt{IPROVER\_HOME} to the
directory that contains the \texttt{iproveropt} executable. iProver depends on
Vampire to clausify problems, so make sure that Vampire is installed as well.

\item[\labelitemi] \textbf{\textit{leo2}:} LEO-II is an automatic
higher-order prover developed by Christoph Benzm\"uller et al.\ \cite{leo2},
with support for the TPTP typed higher-order syntax (TH0). To use LEO-II, set
the environment variable \texttt{LEO2\_HOME} to the directory that contains the
\texttt{leo} executable.

\item[\labelitemi] \textbf{\textit{leo3}:} Leo-III is an automatic
higher-order prover developed by Alexander Steen, Max Wisniewski, Christoph
Benzm\"uller et al.\ \cite{leo3},
with support for the TPTP typed higher-order syntax (TH0). To use Leo-III, set
the environment variable \texttt{LEO3\_HOME} to the directory that contains the
\texttt{leo3} executable.

\item[\labelitemi] \textbf{\textit{satallax}:} Satallax is an automatic
higher-order prover developed by Chad Brown et al.\ \cite{satallax}, with
support for the TPTP typed higher-order syntax (TH0). To use Satallax, set the
environment variable \texttt{SATALLAX\_HOME} to the directory that contains the
\texttt{satallax} executable.

\item[\labelitemi] \textbf{\textit{spass}:} SPASS is a first-order resolution
prover developed by Christoph Weidenbach et al.\ \cite{weidenbach-et-al-2009}.
To use SPASS, set the environment variable \texttt{SPASS\_HOME} to the directory
that contains the \texttt{SPASS} executable and \texttt{SPASS\_VERSION} to the
version number (e.g., ``3.8ds''), or install the prebuilt SPASS package from
\download.

\item[\labelitemi] \textbf{\textit{vampire}:} Vampire is a first-order
resolution prover developed by Andrei Voronkov and his colleagues
\cite{riazanov-voronkov-2002}. To use Vampire, set the environment variable
\texttt{VAMPIRE\_HOME} to the directory that contains the \texttt{vampire}
executable and \texttt{VAMPIRE\_VERSION} to the version number (e.g.,
``4.2.2'').

\item[\labelitemi] \textbf{\textit{verit}:} veriT \cite{bouton-et-al-2009} is an
SMT solver developed by David D\'eharbe, Pascal Fontaine, and their colleagues.
It is designed to produce detailed proofs for reconstruction in proof
assistants. To use veriT, set the environment variable \texttt{ISABELLE\_VERIT}
to the complete path of the executable, including the file name.

\item[\labelitemi] \textbf{\textit{z3}:} Z3 is an SMT solver developed at
Microsoft Research \cite{de-moura-2008}. To use Z3, set the environment variable
\texttt{Z3\_SOLVER} to the complete path of the executable, including the
file name.

\item[\labelitemi] \textbf{\textit{z3\_tptp}:} This version of Z3 pretends to
be an ATP, exploiting Z3's support for the TPTP typed first-order format (TF0).
It is included for experimental purposes. To use it, set the environment
variable \texttt{Z3\_TPTP\_HOME} to the directory that contains the
\texttt{z3\_tptp} executable.

\item[\labelitemi] \textbf{\textit{zipperposition}:} Zipperposition
\cite{cruanes-2014} is a higher-order superposition prover developed by Simon
Cruanes, Petar Vukmirovi\'c, and colleagues. To use Zipperposition, set the
environment variable \texttt{ZIPPERPOSITION\_HOME} to the directory that
contains the \texttt{zipperposition} executable and
\texttt{ZIPPERPOSITION\_VERSION} to the version number (e.g., ``2.0.1'').
\end{enum}

\end{sloppy}

Moreover, the following remote provers are supported:

\begin{enum}
\item[\labelitemi] \textbf{\textit{remote\_agsyhol}:} The remote version of
agsyHOL runs on Geoff Sutcliffe's Miami servers \cite{sutcliffe-2000}.

\item[\labelitemi] \textbf{\textit{remote\_alt\_ergo}:} The remote version of
Alt-Ergo runs on Geoff Sutcliffe's Miami servers \cite{sutcliffe-2000}.

\item[\labelitemi] \textbf{\textit{remote\_e}:} The remote version of E runs
on Geoff Sutcliffe's Miami servers \cite{sutcliffe-2000}.

\item[\labelitemi] \textbf{\textit{remote\_iprover}:} The
remote version of iProver runs on Geoff Sutcliffe's Miami servers
\cite{sutcliffe-2000}.

\item[\labelitemi] \textbf{\textit{remote\_leo2}:} The remote version of LEO-II
runs on Geoff Sutcliffe's Miami servers \cite{sutcliffe-2000}.

\item[\labelitemi] \textbf{\textit{remote\_leo3}:} The remote version of Leo-III
runs on Geoff Sutcliffe's Miami servers \cite{sutcliffe-2000}.

\item[\labelitemi] \textbf{\textit{remote\_waldmeister}:} Waldmeister is a unit
equality prover developed by Hillenbrand et al.\ \cite{waldmeister}. It can be
used to prove universally quantified equations using unconditional equations,
corresponding to the TPTP CNF UEQ division. The remote version of Waldmeister
runs on Geoff Sutcliffe's Miami servers.

\item[\labelitemi] \textbf{\textit{remote\_zipperposition}:} The remote
version of Zipperposition runs on Geoff Sutcliffe's Miami servers.
\end{enum}

By default, Sledgehammer runs a subset of CVC4, E, SPASS, Vampire, veriT, and
Z3 in parallel, either locally or remotely---depending on the number of
processor cores available and on which provers are actually installed. It is
generally desirable to run several provers in parallel.

\opnodefault{prover}{string}
Alias for \textit{provers}.

\optrue{slice}{dont\_slice}
Specifies whether the time allocated to a prover should be sliced into several
segments, each of which has its own set of possibly prover-dependent options.
For SPASS and Vampire, the first slice tries the fast but incomplete
set-of-support (SOS) strategy, whereas the second slice runs without it. For E,
up to three slices are tried, with different weighted search strategies and
number of facts. For SMT solvers, several slices are tried with the same options
each time but fewer and fewer facts. According to benchmarks with a timeout of
30 seconds, slicing is a valuable optimization, and you should probably leave it
enabled unless you are conducting experiments.

\nopagebreak
{\small See also \textit{verbose} (\S\ref{output-format}).}

\optrue{minimize}{dont\_minimize}
Specifies whether the proof minimization tool should be invoked automatically
after proof search.

\nopagebreak
{\small See also \textit{preplay\_timeout} (\S\ref{timeouts})
and \textit{dont\_preplay} (\S\ref{timeouts}).}

\opfalse{spy}{dont\_spy}
Specifies whether Sledgehammer should record statistics in
\texttt{\$ISA\-BELLE\_\allowbreak HOME\_\allowbreak USER/\allowbreak spy\_\allowbreak sledgehammer}.
These statistics can be useful to the developers of Sledgehammer. If you are willing to have your
interactions recorded in the name of science, please enable this feature and send the statistics
file every now and then to the author of this manual (\authoremail).
To change the default value of this option globally, set the environment variable
\texttt{SLEDGEHAMMER\_SPY} to \textit{yes}.

\nopagebreak
{\small See also \textit{debug} (\S\ref{output-format}).}

\opfalse{overlord}{no\_overlord}
Specifies whether Sledgehammer should put its temporary files in
\texttt{\$ISA\-BELLE\_\allowbreak HOME\_\allowbreak USER}, which is useful for
debugging Sledgehammer but also unsafe if several instances of the tool are run
simultaneously. The files are identified by the prefixes \texttt{prob\_} and
\texttt{mash\_}; you may safely remove them after Sledgehammer has run.

\textbf{Warning:} This option is not thread-safe. Use at your own risks.

\nopagebreak
{\small See also \textit{debug} (\S\ref{output-format}).}
\end{enum}


\subsection{Relevance Filter}
\label{relevance-filter}

\begin{enum}
\opdefault{fact\_filter}{string}{smart}
Specifies the relevance filter to use. The following filters are available:

\begin{enum}
\item[\labelitemi] \textbf{\textit{mepo}:}
The traditional memoryless MePo relevance filter.

\item[\labelitemi] \textbf{\textit{mash}:}
The MaSh machine learner. Three learning algorithms are provided:

\begin{enum}
\item[\labelitemi] \textbf{\textit{nb}} is an implementation of naive Bayes.

\item[\labelitemi] \textbf{\textit{knn}} is an implementation of $k$-nearest
neighbors.

\item[\labelitemi] \textbf{\textit{nb\_knn}} (also called \textbf{\textit{yes}}
and \textbf{\textit{sml}}) is a combination of naive Bayes and $k$-nearest
neighbors.
\end{enum}

In addition, the special value \textit{none} is used to disable machine learning by
default (cf.\ \textit{smart} below).

The default algorithm is \textit{nb\_knn}. The algorithm can be selected by
setting the ``MaSh'' option under ``Plugins > Plugin Options > Isabelle >
General'' in Isabelle/jEdit. Persistent data for both algorithms is stored in
the directory \texttt{\$ISABELLE\_\allowbreak HOME\_\allowbreak USER/\allowbreak
mash}.

\item[\labelitemi] \textbf{\textit{mesh}:} The MeSh filter, which combines the
rankings from MePo and MaSh.

\item[\labelitemi] \textbf{\textit{smart}:} A combination of MePo, MaSh, and
MeSh. If the learning algorithm is set to be \textit{none}, \textit{smart}
behaves like MePo.
\end{enum}

\opdefault{max\_facts}{smart\_int}{smart}
Specifies the maximum number of facts that may be returned by the relevance
filter. If the option is set to \textit{smart} (the default), it effectively
takes a value that was empirically found to be appropriate for the prover.
Typical values lie between 50 and 1000.

\opdefault{fact\_thresholds}{float\_pair}{\upshape 0.45~0.85}
Specifies the thresholds above which facts are considered relevant by the
relevance filter. The first threshold is used for the first iteration of the
relevance filter and the second threshold is used for the last iteration (if it
is reached). The effective threshold is quadratically interpolated for the other
iterations. Each threshold ranges from 0 to 1, where 0 means that all theorems
are relevant and 1 only theorems that refer to previously seen constants.

\optrue{learn}{dont\_learn}
Specifies whether Sledgehammer invocations should run MaSh to learn the
available theories (and hence provide more accurate results). Learning takes
place only if MaSh is enabled.

\opdefault{max\_new\_mono\_instances}{int}{smart}
Specifies the maximum number of monomorphic instances to generate beyond
\textit{max\_facts}. The higher this limit is, the more monomorphic instances
are potentially generated. Whether monomorphization takes place depends on the
type encoding used. If the option is set to \textit{smart} (the default), it
takes a value that was empirically found to be appropriate for the prover. For
most provers, this value is 100.

\nopagebreak
{\small See also \textit{type\_enc} (\S\ref{problem-encoding}).}

\opdefault{max\_mono\_iters}{int}{smart}
Specifies the maximum number of iterations for the monomorphization fixpoint
construction. The higher this limit is, the more monomorphic instances are
potentially generated. Whether monomorphization takes place depends on the
type encoding used. If the option is set to \textit{smart} (the default), it
takes a value that was empirically found to be appropriate for the prover.
For most provers, this value is 3.

\nopagebreak
{\small See also \textit{type\_enc} (\S\ref{problem-encoding}).}

\opdefault{induction\_rules}{string}{smart}
Specifies whether induction rules should be considered as relevant facts.
The following behaviors are available:
\begin{enum}
\item[\labelitemi] \textbf{\textit{include}:}
Induction rules are considered by the relevance filter.

\item[\labelitemi] \textbf{\textit{exclude}:}
Induction rules are ignored by the relevance filter.

\item[\labelitemi] \textbf{\textit{instantiate}:}
Induction rules are instantiated based on the conjecture and then
considered by the relevance filter.

\item[\labelitemi] \textbf{\textit{smart}:}
Same as \textit{include} if the prover supports higher-order logic;
same as \textit{exclude} otherwise.
\end{enum}
\end{enum}


\subsection{Problem Encoding}
\label{problem-encoding}

\newcommand\comb[1]{\const{#1}}

\begin{enum}
\opdefault{lam\_trans}{string}{smart}
Specifies the $\lambda$ translation scheme to use in ATP problems. The supported
translation schemes are listed below:

\begin{enum}
\item[\labelitemi] \textbf{\textit{lifting}:} Introduce a new
supercombinator \const{c} for each cluster of $n$~$\lambda$-abstractions,
defined using an equation $\const{c}~x_1~\ldots~x_n = t$ ($\lambda$-lifting).

\item[\labelitemi] \textbf{\textit{opaque\_lifting}:} Same as
\textit{lifting}, except that the supercombinators are kept opaque,
i.e. they are unspecified fresh constants. This effectively disables
all reasoning under $\lambda$-abstractions.

\item[\labelitemi] \textbf{\textit{combs}:} Rewrite lambdas to the Curry
combinators (\comb{I}, \comb{K}, \comb{S}, \comb{B}, \comb{C}). Combinators
enable the ATPs to synthesize $\lambda$-terms but tend to yield bulkier formulas
than $\lambda$-lifting: The translation is quadratic in the worst case, and the
equational definitions of the combinators are very prolific in the context of
resolution.

\item[\labelitemi] \textbf{\textit{opaque\_combs}:} Same as
\textit{combs}, except that the combinators are kept opaque, i.e. without
equational definitions.

\item[\labelitemi] \textbf{\textit{combs\_and\_lifting}:} Introduce a new
supercombinator \const{c} for each cluster of $\lambda$-abstractions and characterize it both using a
lifted equation $\const{c}~x_1~\ldots~x_n = t$ and via Curry combinators.

\item[\labelitemi] \textbf{\textit{combs\_or\_lifting}:} For each cluster of
$\lambda$-abstractions, heuristically choose between $\lambda$-lifting and Curry
combinators.

\item[\labelitemi] \textbf{\textit{keep\_lams}:}
Keep the $\lambda$-abstractions in the generated problems. This is available
only with provers that support $\lambda$s.

\item[\labelitemi] \textbf{\textit{smart}:} The actual translation scheme used
depends on the ATP and should be the most efficient scheme for that ATP.
\end{enum}

For SMT solvers, the $\lambda$ translation scheme is always \textit{lifting},
irrespective of the value of this option.

\opsmartx{uncurried\_aliases}{no\_uncurried\_aliases}
Specifies whether fresh function symbols should be generated as aliases for
applications of curried functions in ATP problems.

\opdefault{type\_enc}{string}{smart}
Specifies the type encoding to use in ATP problems. Some of the type encodings
are unsound, meaning that they can give rise to spurious proofs
(unreconstructible using \textit{metis}). The type encodings are
listed below, with an indication of their soundness in parentheses.
An asterisk (*) indicates that the encoding is slightly incomplete for
reconstruction with \textit{metis}, unless the \textit{strict} option (described
below) is enabled.

\begin{enum}
\item[\labelitemi] \textbf{\textit{erased} (unsound):} No type information is
supplied to the ATP, not even to resolve overloading. Types are simply erased.

\item[\labelitemi] \textbf{\textit{poly\_guards} (sound):} Types are encoded using
a predicate \const{g}$(\tau, t)$ that guards bound
variables. Constants are annotated with their types, supplied as extra
arguments, to resolve overloading.

\item[\labelitemi] \textbf{\textit{poly\_tags} (sound):} Each term and subterm is
tagged with its type using a function $\const{t\/}(\tau, t)$.

\item[\labelitemi] \textbf{\textit{poly\_args} (unsound):}
Like for \textit{poly\_guards} constants are annotated with their types to
resolve overloading, but otherwise no type information is encoded. This
is the default encoding used by the \textit{metis} proof method.

\item[\labelitemi]
\textbf{%
\textit{raw\_mono\_guards}, \textit{raw\_mono\_tags} (sound); \\
\textit{raw\_mono\_args} (unsound):} \\
Similar to \textit{poly\_guards}, \textit{poly\_tags}, and \textit{poly\_args},
respectively, but the problem is additionally monomorphized, meaning that type
variables are instantiated with heuristically chosen ground types.
Monomorphization can simplify reasoning but also leads to larger fact bases,
which can slow down the ATPs.

\item[\labelitemi]
\textbf{%
\textit{mono\_guards}, \textit{mono\_tags} (sound);
\textit{mono\_args} \\ (unsound):} \\
Similar to
\textit{raw\_mono\_guards}, \textit{raw\_mono\_tags}, and
\textit{raw\_mono\_\allowbreak args}, respectively but types are mangled in constant names
instead of being supplied as ground term arguments. The binary predicate
$\const{g}(\tau, t)$ becomes a unary predicate
$\const{g\_}\tau(t)$, and the binary function
$\const{t}(\tau, t)$ becomes a unary function
$\const{t\_}\tau(t)$.

\item[\labelitemi] \textbf{\textit{mono\_native} (sound):} Exploits native
first-order types if the prover supports the TF0, TF1, TH0, or TH1 syntax;
otherwise, falls back on \textit{mono\_guards}. The problem is monomorphized.

\item[\labelitemi] \textbf{\textit{mono\_native\_fool} (sound):} Exploits native
first-order types, including Booleans, if the prover supports the TFX0, TFX1,
TH0, or TH1 syntax; otherwise, falls back on \textit{mono\_native}. The problem
is monomorphized.

\item[\labelitemi] \textbf{\textit{mono\_native\_higher},
\textit{mono\_native\_higher\_fool} \\ (sound):} Exploits native higher-order
types, including Booleans if ending with ``\textit{\_fool}'', if the prover
supports the TH0 syntax; otherwise, falls back on \textit{mono\_native} or
\textit{mono\_native\_fool}. The problem is monomorphized.

\item[\labelitemi] \textbf{\textit{poly\_native}, \textit{poly\_native\_fool},
\textit{poly\_native\_higher}, \\ \textit{poly\_native\_higher\_fool} (sound):}
Exploits native first-order polymorphic types if the prover supports the TF1,
TFX1, or TH1 syntax; otherwise, falls back on \textit{mono\_native},
\textit{mono\_native\_fool}, \textit{mono\_native\_higher}, or
\textit{mono\_native\_higher\_fool}.

\item[\labelitemi]
\textbf{%
\textit{poly\_guards}?, \textit{poly\_tags}?, \textit{raw\_mono\_guards}?, \\
\textit{raw\_mono\_tags}?, \textit{mono\_guards}?, \textit{mono\_tags}?, \\
\textit{mono\_native}? (sound*):} \\
The type encodings \textit{poly\_guards}, \textit{poly\_tags},
\textit{raw\_mono\_guards}, \textit{raw\_mono\_tags}, \textit{mono\_guards},
\textit{mono\_tags}, and \textit{mono\_native} are fully typed and sound. For
each of these, Sledgehammer also provides a lighter variant identified by a
question mark (`\hbox{?}')\ that detects and erases monotonic types, notably
infinite types. (For \textit{mono\_native}, the types are not actually erased
but rather replaced by a shared uniform type of individuals.) As argument to the
\textit{metis} proof method, the question mark is replaced by a
\hbox{``\textit{\_query\/}''} suffix.

\item[\labelitemi]
\textbf{%
\textit{poly\_guards}??, \textit{poly\_tags}??, \textit{raw\_mono\_guards}??, \\
\textit{raw\_mono\_tags}??, \textit{mono\_guards}??, \textit{mono\_tags}?? \\
(sound*):} \\
Even lighter versions of the `\hbox{?}' encodings. As argument to the
\textit{metis} proof method, the `\hbox{??}' suffix is replaced by
\hbox{``\textit{\_query\_query\/}''}.

\item[\labelitemi]
\textbf{%
\textit{poly\_guards}@, \textit{poly\_tags}@, \textit{raw\_mono\_guards}@, \\
\textit{raw\_mono\_tags}@ (sound*):} \\
Alternative versions of the `\hbox{??}' encodings. As argument to the
\textit{metis} proof method, the `\hbox{@}' suffix is replaced by
\hbox{``\textit{\_at\/}''}.

\item[\labelitemi] \textbf{\textit{poly\_args}?, \textit{raw\_mono\_args}? (unsound):} \\
Lighter versions of \textit{poly\_args} and \textit{raw\_mono\_args}.

\item[\labelitemi] \textbf{\textit{smart}:} The actual encoding used depends on
the ATP and should be the most efficient sound encoding for that ATP.
\end{enum}

For SMT solvers, the type encoding is always \textit{mono\_native}, irrespective
of the value of this option.

\nopagebreak
{\small See also \textit{max\_new\_mono\_instances} (\S\ref{relevance-filter})
and \textit{max\_mono\_iters} (\S\ref{relevance-filter}).}

\opfalse{strict}{non\_strict}
Specifies whether Sledgehammer should run in its strict mode. In that mode,
sound type encodings marked with an asterisk (*) above are made complete
for reconstruction with \textit{metis}, at the cost of some clutter in the
generated problems. This option has no effect if \textit{type\_enc} is
deliberately set to an unsound encoding.
\end{enum}


\subsection{Output Format}
\label{output-format}

\begin{enum}

\opfalse{verbose}{quiet}
Specifies whether the \textbf{sledgehammer} command should explain what it does.

\opfalse{debug}{no\_debug}
Specifies whether Sledgehammer should display additional debugging information
beyond what \textit{verbose} already displays. Enabling \textit{debug} also
enables \textit{verbose} behind the scenes.

\nopagebreak
{\small See also \textit{spy} (\S\ref{mode-of-operation}) and
\textit{overlord} (\S\ref{mode-of-operation}).}

\opsmart{isar\_proofs}{no\_isar\_proofs}
Specifies whether Isar proofs should be output in addition to one-line proofs.
The construction of Isar proof is still experimental and may sometimes fail;
however, when they succeed they are usually faster and more intelligible than
one-line proofs. If the option is set to \textit{smart} (the default), Isar
proofs are only generated when no working one-line proof is available.

\opdefault{compress}{int}{smart}
Specifies the granularity of the generated Isar proofs if \textit{isar\_proofs}
is explicitly enabled. A value of $n$ indicates that each Isar proof step should
correspond to a group of up to $n$ consecutive proof steps in the ATP proof. If
the option is set to \textit{smart} (the default), the compression factor is 10
if the \textit{isar\_proofs} option is explicitly enabled; otherwise, it is
$\infty$.

\optrueonly{dont\_compress}
Alias for ``\textit{compress} = 1''.

\optrue{try0}{dont\_try0}
Specifies whether standard proof methods such as \textit{auto} and
\textit{blast} should be tried as alternatives to \textit{metis} in Isar proofs.
The collection of methods is roughly the same as for the \textbf{try0} command.

\optrue{smt\_proofs}{no\_smt\_proofs}
Specifies whether the \textit{smt} proof method should be tried in addition to
Isabelle's built-in proof methods.
\end{enum}


\subsection{Regression Testing}
\label{regression-testing}

\begin{enum}
\opnodefault{expect}{string}
Specifies the expected outcome, which must be one of the following:

\begin{enum}
\item[\labelitemi] \textbf{\textit{some}:} Sledgehammer found a proof.
\item[\labelitemi] \textbf{\textit{none}:} Sledgehammer found no proof.
\item[\labelitemi] \textbf{\textit{timeout}:} Sledgehammer timed out.
\item[\labelitemi] \textbf{\textit{unknown}:} Sledgehammer encountered some
problem.
\end{enum}

Sledgehammer emits an error if the actual outcome differs from the expected outcome. This option is
useful for regression testing.

\nopagebreak
{\small See also \textit{timeout} (\S\ref{timeouts}).}
\end{enum}


\subsection{Timeouts}
\label{timeouts}

\begin{enum}
\opdefault{timeout}{float}{\upshape 30}
Specifies the maximum number of seconds that the automatic provers should spend
searching for a proof. This excludes problem preparation and is a soft limit.

\opdefault{preplay\_timeout}{float}{\upshape 1}
Specifies the maximum number of seconds that \textit{metis} or other proof
methods should spend trying to ``preplay'' the found proof. If this option
is set to 0, no preplaying takes place, and no timing information is displayed
next to the suggested proof method calls.

\nopagebreak
{\small See also \textit{minimize} (\S\ref{mode-of-operation}).}

\optrueonly{dont\_preplay}
Alias for ``\textit{preplay\_timeout} = 0''.

\end{enum}

\section{Mirabelle Testing Tool}
\label{mirabelle}

The \texttt{isabelle mirabelle} tool executes Sledgehammer or other advisory
tools (e.g., Nitpick) or tactics (e.g., \textit{auto}) on all subgoals emerging
in a theory. It is typically used to measure the success rate of a proof tool
on some benchmark. Its command-line usage is as follows:

{\small
\begin{verbatim}
Usage: isabelle mirabelle [OPTIONS] [SESSIONS ...]

  Options are:
    -A ACTION    add to list of actions
    -O DIR       output directory for log files (default: "mirabelle")
    -T THEORY    theory restriction: NAME or NAME[FIRST_LINE:LAST_LINE]
    -m INT       max. no. of calls to each action (0: unbounded) (default 0)
    -s INT       run actions on every nth goal (0: uniform distribution) (default 1)
    -t SECONDS   timeout in seconds for each action (default 30)
    ...

  Apply the given ACTIONs at all theories and proof steps of the
  specified sessions.

  The absence of theory restrictions (option -T) means to check all
  theories fully. Otherwise only the named theories are checked.
\end{verbatim}
}

Option \texttt{-A ACTION} specifies an action to run on all subgoals. When
specified multiple times, all actions are performed in parallel on all
selected subgoals. Available actions are \texttt{arith}, \texttt{metis},
\texttt{quickcheck}, \texttt{sledgehammer}, \texttt{sledgehammer\_filter}, and
\texttt{try0}.

Option \texttt{-O DIR} specifies the output directory, which is created
if needed. In this directory, a log file named "mirabelle.log" records the
position of each tested subgoal and the result of executing the actions.

Option \texttt{-T THEORY} restricts the subgoals to those emerging from this
theory. When not provided, all subgoals from are theories are selected. When
provided multiple times, the union of all specified theories' subgoals is
selected.

Option \texttt{-m INT} specifies a maximum number of goals on which the action
are run.

Option \texttt{-s INT} specifies a stride, effectively running the actions on
every n$^{th}$ goal.

Option \texttt{-t SECONDS} specifies a generic timeout that the actions may
interpret differently.

More specific documentation about low-level options, the \texttt{ACTION}
parameter, and its corresponding options can be found in the Isabelle tool
usage by entering \texttt{isabelle mirabelle -?} on the command line.

The following subsections assume that the environment variable \texttt{AFP} is
defined and points to a release of the Archive of Formal Proofs.

\subsection{Example of Benchmarking Sledgehammer}

\begin{verbatim}
isabelle mirabelle -d $AFP -O output \
  -A "sledgehammer[provers = e, timeout = 30]" \
  VeriComp
\end{verbatim}

This command specifies to run the Sledgehammer action, using the E prover with
a timeout of 30 seconds, on all subgoals emerging from all theory in the AFP
session VeriComp. The results are written to \texttt{output/mirabelle.log}.

\begin{verbatim}
isabelle mirabelle -d $AFP -O output \
  -T Semantics -T Compiler \
  -A "sledgehammer[provers = e, timeout = 30]" \
  VeriComp
\end{verbatim}

This command also specifies to run the Sledgehammer action, but this time only
on subgoals emerging from theories Semantics or Compiler.


\subsection{Example of Benchmarking Multiple Tools}

\begin{verbatim}
isabelle mirabelle -d $AFP -O output -t 10 \
  -A "try0" -A "metis" \
  VeriComp
\end{verbatim}

This command specifies two actions running the \textbf{try0} and \textbf{metis}
commands, respectively, each with a timeout of 10 seconds. The results are
written to \texttt{output/mirabelle.log}.

\subsection{Example of Generating TPTP Files}

\begin{verbatim}
isabelle mirabelle -d $AFP -O output \
  -A "sledgehammer[provers = e, timeout = 5, keep = true]" \
  VeriComp
\end{verbatim}

This command generates TPTP files using Sledgehammer. Since the file
is generated at the very beginning of every Sledgehammer invocation,
a timeout of five seconds making the prover fail faster speeds up
processing the subgoals. The results are written in an action-specific
subdirectory of the specified output directory (\texttt{output}). A TPTP
file is generated for each subgoal.

\let\em=\sl
\bibliography{manual}{}
\bibliographystyle{abbrv}

\end{document}
